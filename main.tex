\documentclass{article}
% Description: This file contains the preamble for the resume. It contains the packages and commands that are used in the resume.

% Packages
\usepackage{xparse}
\usepackage[margin=1.3cm]{geometry}
\usepackage[hidelinks]{hyperref}
\usepackage{environ}
\usepackage{newtxtext,newtxmath}
\usepackage{anyfontsize}
\usepackage{enumitem}
\usepackage{comment}
\usepackage{setspace}

\setstretch{1.2}

% Custom Commands
% Education Section
\newcommand{\horizontal}{\vspace{2pt}\hrule}
\newcommand{\school}[7]{
  \vspace{2pt}
  \textsc{\textbf{#1}} \hfill \textbf{#2} \\ 
  \textbf{#3} \hfill \textit{#4} \\ 
  \textbf{#5} $\mid$ #6 GPA \hfill \textit{#7} \\
}
% New Section
\newcommand{\sectitle}[1]{\vspace{3pt} \textbf{\large \MakeUppercase{#1}} \horizontal \vspace{3pt}}
% Skill Section
\newcommand{\skill}[2]{\textbf{#1:} #2}

% Custom Environments
\NewDocumentEnvironment{experience}{mmmm}
    {
        \vspace{2pt}
        \textsc{\textbf{\large #1}} \hfill \textbf{#2} \\ % Added line break for better alignment
        \textbf{#3} \hfill \textit{#4} \\ % Improved alignment and formatting
        \begin{itemize}[noitemsep,topsep=0pt]
    }
    {
        \end{itemize}
    }

\NewDocumentEnvironment{subexperience}{mm}
    {
        \vspace{2pt}
        \textit{\textbf{#1}} \hfill \textit{\textbf{#2}} \\ % Added line break for better readability
        \begin{itemize}[noitemsep,topsep=0pt]
    }
    {
        \end{itemize}
    }

\NewDocumentEnvironment{project}{mmmm}
    {
        \vspace{2pt}
        \textsc{\textbf{\large #1}}{#2} $|$ \textit{#3} \hfill \textit{#4} \\
        \begin{itemize}[noitemsep,topsep=0pt]
    }
    {
        \end{itemize}
    }

\NewDocumentEnvironment{experiencenolist}{mmmm}
    {
        \vspace{2pt}
        \textsc{\textbf{\large #1}} \hfill \textbf{#2} \\ % Added line break for better alignment
        \textbf{#3} \hfill \textit{#4} \\ % Improved formatting
    }{}


\begin{document}

\thispagestyle{empty}

\begin{center}
    \textbf{\LARGE Alexander Yoon} \\
    \href{mailto:"ayoon37@gatech.edu"}{ayoon37@gatech.edu} $|$ (470) 428-1484 $|$ \href{https://www.linkedin.com/in/alexander-yoon/}{https://www.linkedin.com/in/alexander-yoon/} $|$ \href{https://github.com/XanderYoon}{https://github.com/XanderYoon}\\
\end{center}

\begin{flushleft}

\sectitle{EDUCATION}

\school{Georgia Institute of Technology}{Atlanta, Georgia}
{Master of Science in Computer Science: Machine Learning}{May 2026}
{Bachelor of Science in Computer Science}{ 3.81} {May 2025}
{Relevant Coursework: Data Structures \& Algorithms, Machine Learning, Design \& Analysis of Algorithms, Probability \& Statistics, Combinatorics, Objects \& Design, Object-Orientated Programming}

\sectitle{EXPERIENCE}

    \begin{experience}{GEORGIA TECH, EDX LINEAR ALGEBRA COURSE}{Atlanta, GA}{Curriculum Developer}{January 2023 -- Present}
        \item Developed additional practice quizzes complete with comprehensive solutions for the Georgia Tech EdX Linear Algebra course.
        \item Enhanced a four-part Linear Algebra course through meticulous analysis, editing, and refinement, elevating its efficacy as an educational asset.
        \item Conducted in-depth research on other EdX courses to benchmark and integrate successful strategies into our curriculum, benefiting our 4,000 enrolled students.
        \item Leveraged large language models to anticipate and prevent potential pitfalls and distractors in problem-solving for enhanced student learning outcomes.
    \end{experience}

    \begin{experience}{GEORGIA TECH, MEDFORD RESEARCH LAB}{Atlanta, GA}{Undergraduate Research Assistant}{August 2023 -- December 2023}
        \item Effectively utilized Linux on Georgia Tech's supercomputer infrastructure to process and analyze high-dimensional datasets of up to 200 features. 
        \item Implemented LASSO and L1 regularization to perform dimensionality reduction on infrared spectrum data for ethanol analysis which allowed for our LDA model to converge. 
        \item Utilized Python's ASE library to generate atomic structures and subsequently employed Quantum Espresso for multi-faceted calculations including energy, forces, density, and electrostatic potential, allowing us to calculate the change of energy when an adsorbate is attached to the surface of an adsorbent.
    \end{experience}

\sectitle{PROJECTS}

    \vspace{3pt}

    \begin{project}{ART VALUATION PREDICTION}{}{Python, BeatufiulSoup, Tableau, Catboost}{January 2024}
        \item Employed BeautifulSoup to scrape and analyze sentiment from over 30 art market articles spanning a decade, utilizing NLTK for sentiment analysis. The derived metrics were integrated as additional features to enrich our model's training data.
        \item Utilized polynomial regression techniques to generate predictive trend-lines for art piece market prices, ensuring model accuracy by mitigating overfitting and underfitting.
        \item Implemented a gradient boosting model to estimate current valuations, achieving a remarkable Root Mean Square Error (RMSE) of 0.07 despite working with a limited dataset.
    \end{project}
    
    \begin{project}{MACHINE LEARNING SOCCER PREDICTION}{}{Python, sklearn, PyTorch, NumPy, Matplotlib}{August -- December 2023}
        \item Worked on a team of 5 to build and train logistic regression, random forest, and artificial neural network models using Scikit-Learn and PyTorch to predict soccer match outcomes with 70\% accuracy, beating benchmark betting odds data by 8\%.
        \item Developed feature engineering strategies, performed dimensionality reduction, and conducted hyperparameter tuning to reduce overfitting, improving model accuracy by $\sim$10\%.
    \end{project}

    \begin{project}{3RD INFANTRY DIVISION TANK DECOY}{}{C++, Arduino (ESP8266)}{January 2023}
        \item Developed compact radio-emitting devices designed as tank decoys, strategically exploiting the electromagnetic spectrum to provide our military with the tactical advantage.
        \item Established a client-server configuration enabling simultaneous control of multiple decoys, enhancing operational flexibility and effectiveness.
        \item Optimized decoy performance by integrating a deep sleep feature, significantly extending the device lifespan by over twofold.
        \item Implemented random relay cycling to simulate frequency hopping, augmenting the decoy's deceptive capabilities and operational resilience.
    \end{project}

    \begin{project}{ROTC FTX PERFORMANCE ANALYSIS}{}{Excel, Pandas, Seaborn, Matplotlib, Jupyter Notebook}{August 2022}
        \item Utilized Python's Pandas library to efficiently process and format Battalion performance data collected during the Field Training Exercise (FTX).
        \item Conducted thorough data analysis, uncovering notable grading bias among particular instructors towards specific markers.
        \item Leveraged Seaborn and Matplotlib to create compelling data visualizations, facilitating the presentation and validation of research findings.
    \end{project}
    
\sectitle{SKILLS}

    \vspace{3pt}
    \skill{Technologies}{Java (Android Studio), Python (NumPy, sklearn, BeautifulSoup), SQL, Tableau, Excel, Git, Linux, Bash, LaTeX} \\
    \skill{Involvements}{Boxing Club, Big Data Big Impact, Wreck Camp, Hackathons, Delta Chi Fraternity}

\end{flushleft}

\end{document}